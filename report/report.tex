\documentclass[12pt]{article}

\usepackage[margin=1.25in]{geometry}
\usepackage{amsmath, amssymb}
\usepackage{graphicx}
\usepackage{float}
\usepackage{enumitem}
\usepackage[backend=biber]{biblatex}

\setlength{\parindent}{10pt}

\addbibresource{literature.bib}

\title{Segmentation of satellite images with U-Net}
\date{26.01.2022}
\author{Petrus Salminen}

\begin{document}
\begin{titlepage}
	\centering
	\vspace*{2.5cm}
	{\huge\bfseries Segmentation of satellite images with U-Net\par}
	\vspace{0.5cm}
	{\large Advanced Machine Learning (SS 2022)\par
	Daniel Richter, Petrus Salminen\par}
	\vfill
	{\large \today\par}
\end{titlepage}


\newpage
\tableofcontents
\newpage

\section{Introduction}

Image segmentation is the process of categorizing an image on the pixel level into multiple regions. These regions result in a simplification of the original image which can then be used for further analysis. Typical examples for image segmentation problems are:
\begin{itemize}
	\item Recognizing streets, buildings, cars, pedestrians etc. in real-time on videos. Such methods are utilized especially in the fields of robotics and autonomous driving.
	\item Finding abnormal formations on medical images. The resulting segmentations can help doctors to find outbreaking diseases or other severe anomalies earlier and easier than by interpreting the images themselves.
	\item Creating regions to represent fields, forests, buildings etc. on satellite images. This is especially useful in analyzing scenery changes or automatic landscape generation (e.g. for video games or scientific simulations).
\end{itemize}
There are three different types of image segmentation:
\begin{itemize}
	\item Semantic segmentation: Categorize every pixel without differentiation of distinct instances of categories (e.g. background, road, person).
	\item Instance segmentation: Categorize only relevant pixels with differentiation of distinct instances categories (e.g. person1, person2, car1, car2).
	\item Panoptic segmentation: Categorize every pixel with differentiation of distinct instances. This is basically a combination of semantic and instance segmentation.
\end{itemize}
One of the classical deep learning approaches to this problem is the U-Net-architecture first introduced in Ronneberger et al. (2015). For the final project we will also use this approach for the segmentation of satellite images. \newline
In this report we will first present the data structure and sources. In the next section we will go over the exact implementation and evaluation of our U-Net. In the section chapter we will discuss the results of the training with different variations of parameters. In the last section we will describe extensions of the U-Net architecture and discuss more sophisticated methods which are currently used in the industry.
\newpage

\section{Data}
2 Seiten
\newpage

\section{Methods and implementation}
4 Seiten
\newpage

\section{Results}
6-7 Seiten
\newpage

\section{Further extensions of U-Net}
3 Seiten
\newpage

\section{Other methods}
3 Seiten
\newpage


\newpage
\printbibliography
\end{document}